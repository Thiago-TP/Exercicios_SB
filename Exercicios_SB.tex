\documentclass[11pt, a4paper]{book}

% Pacotes padrões para lidar com palavras em português
\usepackage[T1]{fontenc}    % https://ctan.org/pkg/fontenc?lang=en
\usepackage[utf8]{inputenc} % https://ctan.org/pkg/inputenc?lang=en
\usepackage[brazil]{babel}  % https://ctan.org/pkg/babel?lang=en

% Mudança da formatação das páginas
\usepackage[margin=2.5cm]{geometry} % https://ctan.org/geometry/pgf?lang=en
%\usepackage{fancyhdr}

% Pacote para desenhos
\usepackage{tikz} % https://ctan.org/pkg/pgf?lang=en

% Mudança da estética dos títulos de capítulo, seção, e a subseção
\usepackage[scaled]{helvet} % https://ctan.org/pkg/helvet?lang=en
\usepackage{titlesec}       % https://ctan.org/pkg/titlesec?lang=en
	\titleformat{\chapter}
	{\sffamily\huge\bfseries\color{cyan!60!black}}
	{Parte \thechapter\ --}{5pt}{}
	
	\titleformat{\section}
	{\normalfont\sffamily\LARGE\bfseries\color{cyan!50!black}}
	{\thesection\ }{0pt}{}
	
	\titleformat{\subsection}
	{\normalfont\sffamily\Large\bfseries\color{cyan!25!black}}
	{}{0pt}{}

% Pacote enumitem permite colocar número da seção na enumeração, e introduz a opção resume.
% https://tex.stackexchange.com/questions/1126/include-section-number-in-list-number
% https://tex.stackexchange.com/questions/351638/continue-numbering-of-paragraphs-after-a-new-section
\usepackage{enumitem}
    \setenumerate[1]{label=\thesection.\arabic*.}
    \setenumerate[2]{label*=\arabic*.}

% Atalho para escrever uma linha de código.
% Uso: \asm{linha de código}
% Na linha, os símbolos _ devem ser reescritos como \_
\newcommand{\asm}[1]{{\tt#1}}


% Pacote para inserção de códigos
\usepackage[newfloat, section]{minted} % https://ctan.org/pkg/minted?lang=en
% \usemintedstyle{...} % caso queira mudar o esquema de cores, troque ... por uma das opções em "pygmentize -L styles"

% Atalho para inserir código NASM x86 no pdf.
% Uso: \putNASM{legenda}{nome do arquivo}{label para referência}
% O arquivo deve estar na pasta 'listings'
\usepackage{float}
\newcommand{\putNASM}[3]{
    \begin{listing}[H]
        \caption{#1}
        \inputminted[linenos=true, frame=single]{nasm}{listings/#2}
        \label{#3}
    \end{listing}
}


\begin{document}

\chapter{Exercícios de Software Básico}

\section{Módulo 1 -- Compiladores}
\subsection{Questões Teóricas}			
\subsection{Questões Práticas}

\section{Módulo 2 -- Assembly x86-64}
\subsection{Questões Teóricas}

\begin{enumerate}
    \item
    Os itens abaixo possuem instruções de programas Assembly IA-32 (em modo nativo) 
    que utilizam diversos modos de endereçamento. 
    Classifique cada item como correto ou errado, e justifique o que estiver errado.
    \begin{itemize}
        \item[(a)] \asm{mov EAX, 10}
        \item[(b)] \asm{mov [M], AL}
        \item[(c)] \asm{mov AL, [CS + ESI + array]}
        \item[(d)] \asm{mov vetor[1], 0}
        \item[(e)] \asm{add AX, [X + ECX]}
        \item[(f)] \asm{mov ESI, vetor + EBX}
        \item[(g)] \asm{inc WORD [inicio + EBX*8 + ESI]}
        \item[(h)] \asm{mov [EBX + ESI*4], DWORD 5}
        \item[(i)] \asm{dec BYTE [BL]}
        \item[(j)] \asm{add [x + 1], AL} 
        \item[(k)] \asm{mov EAX, [array + ECX*8 + EBX]}
        \item[(l)] \asm{mov [EAX*8 + 1], 5}
        \item[(m)] \asm{mov BL, AX}
        \item[(n)] \asm{cmp [ESI], 10}
        \item[(o)] \asm{adc AL, AH}
    \end{itemize}

    \item
\end{enumerate}
			
\subsection{Questões Práticas}
\begin{enumerate}[resume]
\item
O programa abaixo realiza a cópia de um vetor de \textit{double words}, 
convertendo-o de \textit{little endian} para \textit{big endian}.
Complete o programa, indicando as instruções dos espaços em branco 
(cada espaço deve ser preenchido com uma única instrução).


\putNASM{código bala}{little_endian.asm}{}
%\begin{listing}[h]
%    \caption{código bala}
%    \inputminted[linenos=true, frame=single]{nasm}{listings/little_endian.asm}
%    \label{lst:little_endian}
%\end{listing}


\end{enumerate}
		
%\chapter{Respostas aos exercícios}
%\section{Módulo 1 -- Compiladores}
%\subsection{Questões Teóricas}
%\subsection{Questões Práticas}

%\section{Módulo 2 -- Assembly x86-64}
%\subsection{Questões Teóricas}
%\subsection{Questões Práticas}

\end{document}
