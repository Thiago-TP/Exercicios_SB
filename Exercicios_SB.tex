\documentclass[11pt, a4paper]{book}

% Pacotes padrões para lidar com palavras em português
\usepackage[T1]{fontenc}    % https://ctan.org/pkg/fontenc?lang=en
\usepackage[utf8]{inputenc} % https://ctan.org/pkg/inputenc?lang=en
\usepackage[brazil]{babel}  % https://ctan.org/pkg/babel?lang=en

% Mudança da formatação das páginas
\usepackage[margin=2.5cm]{geometry} % https://ctan.org/geometry/pgf?lang=en
%\usepackage{fancyhdr}

% Pacote para desenhos
\usepackage{tikz} % https://ctan.org/pkg/pgf?lang=en
\usepackage{multicol}
\usepackage{float}
% Mudança da estética dos títulos de capítulo, seção, e a subseção
\usepackage[scaled]{helvet} % https://ctan.org/pkg/helvet?lang=en
\usepackage{titlesec}       % https://ctan.org/pkg/titlesec?lang=en
	\titleformat{\chapter}
	{\sffamily\huge\bfseries\color{cyan!60!black}}
	{Parte \thechapter\ --}{5pt}{}
	
	\titleformat{\section}
	{\normalfont\sffamily\LARGE\bfseries\color{cyan!50!black}}
	{\thesection\ }{0pt}{}
	
	\titleformat{\subsection}
	{\normalfont\sffamily\Large\bfseries\color{cyan!25!black}}
	{}{0pt}{}

% Pacote enumitem permite colocar número da seção na enumeração, e introduz a opção resume.
% https://tex.stackexchange.com/questions/1126/include-section-number-in-list-number
% https://tex.stackexchange.com/questions/351638/continue-numbering-of-paragraphs-after-a-new-section
\usepackage{enumitem} % https://ctan.org/pkg/enumitem?lang=en
    \setlength{\parindent}{0em}
    \setlist{noitemsep, leftmargin=0em}
    \setenumerate[1]{label=\thesection.\arabic*.}
    \setenumerate[2]{label*=\arabic*.}

% Atalho para escrever uma linha de código.
% Uso: \asm{linha de código}
% Na linha, os símbolos _ devem ser reescritos como \_
\newcommand{\asm}[1]{{\tt#1}}


% Pacote para inserção de códigos
\usepackage[section]{minted} % https://ctan.org/pkg/minted?lang=en
    \definecolor{bg}{rgb}{0.97, 0.97, 0.97}
    \usemintedstyle{default} % caso queira mudar o esquema de cores, troque por uma das opções em "pygmentize -L styles"

% Atalho para inserir código NASM x86 no pdf.
% Uso: \putNASM{legenda}{nome do arquivo}{label para referência}
% O arquivo deve estar na pasta 'listings'
\newcommand{\putNASM}[1]{
    \inputminted[numbers=right, numbersep=1pt, bgcolor=bg]{nasm}{listings/#1}
}
\newcommand{\putC}[1]{
    \inputminted[numbers=right, numbersep=1pt, bgcolor=bg]{C}{listings/#1}
}


\begin{document}

\chapter{Exercícios de Software Básico}
    \section{Módulo 1 -- Compiladores}
        \subsection{Questões Teóricas}
        \begin{enumerate}

    \item 
    Responda sucintamente:
    \begin{itemize}
        \item [(a)]
        Qual a diferença mais importante entre uma macro 
        e uma subrotina (função)?

        \item [(b)]
        Durante o processo de tradução são necessários dois estágios,
        Análise e Síntese. Explique brevemente o objetivo de cada estágio
        e liste as sub-etapas de cada um deles.        
    \end{itemize}

    \item 
    Os itens abaixo podem conter erros.
    Destaque-os, e corrija-os reescrevendo os itens.
    \begin{itemize}
        \item [(a)]
        O bootstrap loader faz parte do Sistema Operacional.
        Ele é carregado no processo de BOOT.
        Possui um tamanho sempre múltiplo a 512 bytes 
        e foi substituído pela GPT para ocupar muito espaço em memória RAM.

        \item [(b)]
        As bibliotecas dinâmicas com carregador estático 
        geram um executável chamado de standalone que tem como vantagem
        ser mais portável que o executável gerado usando 
        bibliotecas dinâmicas com carregador dinâmico.
        Porém, o carregador dinâmico é o único que garante que 
        somente tenha uma única cópia da biblioteca em memória.

        \item [(c)]
        O formato de arquivos .COM é um formato sem cabeçalho 
        de no máximo 64 kB que não permite ligação nem debug.
        É um arquivo tanto para objeto como executável.
        O formato COFF também é formato de objeto executável,
        porém ele permite bibliotecas dinâmicas e debug.

        \item [(d)]
        A principal vantagem de uma linguagem compilada em relação
        a uma interpretada é a otimização completa do código 
        sem precisar manter relação de linha de código compilado
        com o arquivo de texto de entrada com o programa original.
        Porém, uma outra vantagem é que os programas compilados
        são mais portáveis, já que o arquivo executável de um programa
        compilado pode ser executado em qualquer Sistema Operacional.
    \end{itemize}

    \item
    Abaixo estão listadas várias afirmativas incorretas.
    Justificando, identifique os erros, e corrija-os.
    \begin{itemize}
        \item [(a)]
        O formato .COM caracteriza-se por ter um endereço fixo (100H)
        para o ponto de entrada do programa.
        O cabeçalho de um arquivo nesse formato possui tamanho reduzido simplesmente
        composto pelos caracteres `MZ' e 
        a quantidade de segmentos de 64 kiB necessários para esse programa.

        \item [(b)]
        Bootstrap loader é um carregador especial, já que ele consiste em um programa
        armazenado completamente em um único setor conhecido como MBR.

        \item [(c)]
        A utilização de bibliotecas dinâmicas permite que um trecho de código 
        chamado por vários problemas possa ter uma única cópia em memória,
        e somente carregada ao ser executada: 
        o montador e o ligador portanto não necessitam serem informados 
        sobre o uso de uma biblioteca dinâmica.

        \item [(d)]
        O formato ELF é um formato de exclusivo de arquivos objeto complexo 
        que armazena tabelas com informações de realocação para o ligador.
        O formato PE (portable executable) é um formato feito em base no formato ELF.
    \end{itemize} 

    \item 
    Abaixo estão listadas várias afirmativas, as quais podem estar erradas.
    Identifique o(s) erro(s), dê o por quê(s), e corrija-o(s).
    \begin{itemize}
        \item [(a)]
        Executável ligado com biblioteca estática 
        possui a vantagem de ser mais portável
        que o executável com biblioteca dinâmica.
        O programa também é carregado mais rápido em memória.
        Porém, é possível ter várias cópias da mesma biblioteca em memória.
        Por esse motivo os formatos de arquivos mais recentes 
        como ELF e PE não permitem ligação estática.

        \item [(b)]
        A utilização de bibliotecas dinâmicas com carregador estático 
        permite que um trecho de código chamado por vários programas 
        possa ter uma única cópia em memória, e somente carregada ao ser executada.
        O montador e o ligador portanto não necessitam 
        serem informados sobre o uso de uma biblioteca dinâmica. 
    \end{itemize}

    \item
    Descreva as informações contidas no MBR, indicando suas partes.
    Indique qual é o objetivo do setor MBR no processo de carregação do Sistema Operacional.
    Indique onde se encontra o MBR.
    Descreva por que o MBR foi substituído pelo GPT 
    em sistemas computacionais modernos.

    \item
    Explique para que serve o código de 3 endereços
    durante o processo de compilação;
    se é usado durante a fase de análise ou síntese e;
    qual a diferença entre a tabela de 3 colunas e a de 4.

    \item
    Dado o código de três endereços abaixo, 
    que trabalha com array de bytes em memória,
    responda:    
    \putC{3_addresses.c}

    \begin{itemize}
        \item [(a)]
        O código foi otimizado? Justifique.
        
        \item [(b)]
        Assumindo que o código inicial era ANSI C, 
        qual o tipo da variável c2 e
        qual o tipo da variável c1? Jusifique.         
    \end{itemize}

    \item
    Cada um dos seguintes trechos de código em linguagem C 
    possui um ou mais erros. Indique onde estão os erros 
    e classifique-os como léxico, sintático ou semântico.
    \begin{itemize}
        \item [(a)] \putC{for_ab.c}
        \item [(b)] \putC{double_xy.c}
        \item [(c)] \putC{switch_a.c}
        \item [(d)] \putC{float_abc.c}
    \end{itemize}

    \item
    As funções em C abaixo fazem parte do mesmo programa,
    como um arquivo de cabeçalho (*.h). O arquivo não compila.
    Indique as linhas erradas, explicitando se o erro é
    léxico, sintático ou semântico.
    Assuma que CHAR\_BIT foi definido corretamente.
    \putC{h_funcs.c}

    \item
    Detecte os erros no código abaixo, sublinhando o erro
    e indicando se o erro é sintático, léxico ou semântico.
    \putC{print_plus_n.c}

    \item
    No código abaixo identifique os erros
    e indique se é erro léxico, semântico ou sintático.
    \putC{mnpq.c}

    \item
    Indique quais são os erros no código abaixo, classificando-o como
    léxico, sintático ou semântico.
    \putC{wrong_array.c}

\end{enumerate}


        \subsection{Questões Práticas}
        \begin{enumerate}[resume]
    \item
    Considere os módulos a seguir na Linguagem de Montagem Hipotética apresentada em sala.
    
    \begin{multicols}{2}
        \putNASM{Código fonte do módulo A.}{modA.asm}{modA}
        \putNASM{Código fonte do módulo B.}{modB.asm}{modB}
    \end{multicols}

    \begin{itemize}
        \item [(a)]
        Personifique um montador e monte os módulos como uma sequência de números inteiros,
        utilizando a tabela em apêndice como referência. 
        Apresenta o código montado e as tabelas resultantes.

        \item [(b)]
        Personifique um ligador e combine os módulos em um único arquivo executável.
        Apresente o código ligado indicando os endereços absolutos e relativos,
        escrevendo no cabeçalho antes do código um mapa de bits mediante a diretiva R
        (ex.: R 0101010001). 
    \end{itemize}
\end{enumerate}


    \section{Módulo 2 -- Assembly x86-64}
        \subsection{Questões Teóricas}
        \begin{enumerate}
    \item
    Os itens abaixo possuem instruções de programas Assembly IA-32 (em modo nativo) 
    que utilizam diversos modos de endereçamento. 
    Classifique cada item como correto ou errado, e justifique o que estiver errado.
    \begin{itemize}
        \item[(a)] \asm{mov EAX, 10}
        \item[(b)] \asm{mov [M], AL}
        \item[(c)] \asm{mov AL, [CS + ESI + array]}
        \item[(d)] \asm{mov vetor[1], 0}
        \item[(e)] \asm{add AX, [X + ECX]}
        \item[(f)] \asm{mov ESI, vetor + EBX}
        \item[(g)] \asm{inc WORD [inicio + EBX*8 + ESI]}
        \item[(h)] \asm{mov [EBX + ESI*4], DWORD 5}
        \item[(i)] \asm{dec BYTE [BL]}
        \item[(j)] \asm{add [x + 1], AL} 
        \item[(k)] \asm{mov EAX, [array + ECX*8 + EBX]}
        \item[(l)] \asm{mov [EAX*8 + 1], 5}
        \item[(m)] \asm{mov BL, AX}
        \item[(n)] \asm{cmp [ESI], 10}
        \item[(o)] \asm{adc AL, AH}
    \end{itemize}

    \item
    Descreva as diferenças sobre endereçamento e utilização de meória 
    do Modo Real e o Modo Protegido. 
    Para o modo real, indique somente como é calculado o endereço de memória.
    No modo protegido, 
    faça um diagrama mostrando os diferentes segmentos de memória,
    indique como é calculado o endereço lógico e real, 
    e por que o modo é chamado de protegido.

    \item 
    Dadas as seguintes instruções e os seus respectivos códigos de máquina,
    indique os valores dos campos OPCODE, Mod R/M, SIB, DISPLACEMENT, e IMMEDIATE.
    Note que uma instrução pode deixar de apresentar algum campo.
    \begin{itemize}
        \item \asm{MOV edx, 0x0} ba 00 00 00 00
        \item \asm{MOV EBP, ESP} 89 e5
    \end{itemize}

    \item
    Existem três tipos básicos de operandos: imediato, registrador e memória.
    O acesso a memória pode ser feito de duas maneiras: direta ou indireta.
    Em cada instrução com algum tipo de endereçamento 
    do código abaixo especifique que tipo de operando 
    está sendo usado como fonte e destino:
    imediato, memória direta/indireta, ou registrador.
    Indicar se algum endereçamento é ilegal. 

    \putNASM{Código com vários tipos de acesso a memória.}{enderecamentos.asm}{enderecamentos}

\end{enumerate}

			
        \subsection{Questões Práticas}
        \begin{enumerate}[resume]
    \item
    O programa abaixo realiza a cópia de um vetor de \textit{double words}, 
    convertendo-o de \textit{little endian} para \textit{big endian}.
    Complete o programa, indicando as instruções dos espaços em branco 
    (cada espaço deve ser preenchido com uma única instrução).

    \putNASM{Conversor de \textit{little endian} para \textit{big endian}.}{little_endian.asm}{little_endian}

    \item
    Para cada código C abaixo, escreva o equivalente em Assembly IA-32.
    Diretivas em C \textbf{devem} ser substituídas por diretivas equivalentes em IA-32.
    Use os registradores para as variáveis locais (com exceção de estruturas de dados)
    e seção de Dados ou BSS para as variáveis estáticas ou globais.
    \textbf{Deve-se} utilizar os endereçamentos corretos 
    para cada tipo de estrutura de dados.
    Não se preocupe pelo fato do programa principal em C ser uma função.
    \begin{itemize}
        \item [(a)] \putC{Vetor de 100 inteiros.}{vector_size_100.c}{vector_size_100}
        \item [(b)] \putC{Matriz indicadora de inteiros iguais.}{matrix_matching.c}{matrix_matching}
        \item [(c)] \putC{Soma dos elementos de um vetor.}{vector_sum.c}{vector_sum}
        \item [(d)] Não é permitido \asm{MUL} ou \asm{IMUL}.
                    \putC{Preenchimento de matriz 100x100.}{100x100_matrix.c}{100x100_martix}
        \item [(e)] Assuma que o usuário vai digitar um número de 0 a 9.
                    \putC{Soma com parâmetro na entrada do usuário.}{weird_sum.c}{weird_sum}
    \end{itemize}
\end{enumerate}

		
%\chapter{Respostas aos exercícios}
%\section{Módulo 1 -- Compiladores}
%\subsection{Questões Teóricas}
%\subsection{Questões Práticas}

%\section{Módulo 2 -- Assembly x86-64}
%\subsection{Questões Teóricas}
%\subsection{Questões Práticas}

\end{document}
