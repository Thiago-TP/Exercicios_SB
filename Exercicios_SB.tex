\documentclass[11pt, a4paper]{book}

% Pacotes padrões para lidar com palavras em português
\usepackage[T1]{fontenc}    % https://ctan.org/pkg/fontenc?lang=en
\usepackage[utf8]{inputenc} % https://ctan.org/pkg/inputenc?lang=en
\usepackage[brazil]{babel}  % https://ctan.org/pkg/babel?lang=en

% Mudança da formatação das páginas
\usepackage[margin=2.5cm]{geometry} % https://ctan.org/geometry/pgf?lang=en
%\usepackage{fancyhdr}

% Pacote para desenhos
\usepackage{tikz} % https://ctan.org/pkg/pgf?lang=en
\usepackage{multicol}

% Mudança da estética dos títulos de capítulo, seção, e a subseção
\usepackage[scaled]{helvet} % https://ctan.org/pkg/helvet?lang=en
\usepackage{titlesec}       % https://ctan.org/pkg/titlesec?lang=en
	\titleformat{\chapter}
	{\sffamily\huge\bfseries\color{cyan!60!black}}
	{Parte \thechapter\ --}{5pt}{}
	
	\titleformat{\section}
	{\normalfont\sffamily\LARGE\bfseries\color{cyan!50!black}}
	{\thesection\ }{0pt}{}
	
	\titleformat{\subsection}
	{\normalfont\sffamily\Large\bfseries\color{cyan!25!black}}
	{}{0pt}{}

% Pacote enumitem permite colocar número da seção na enumeração, e introduz a opção resume.
% https://tex.stackexchange.com/questions/1126/include-section-number-in-list-number
% https://tex.stackexchange.com/questions/351638/continue-numbering-of-paragraphs-after-a-new-section
\usepackage{enumitem}
    \setlist{noitemsep}
    \setenumerate[1]{label=\thesection.\arabic*.}
    \setenumerate[2]{label*=\arabic*.}

% Atalho para escrever uma linha de código.
% Uso: \asm{linha de código}
% Na linha, os símbolos _ devem ser reescritos como \_
\newcommand{\asm}[1]{{\tt#1}}


% Pacote para inserção de códigos
\usepackage[newfloat, section]{minted} % https://ctan.org/pkg/minted?lang=en
% \usemintedstyle{...} % caso queira mudar o esquema de cores, troque ... por uma das opções em "pygmentize -L styles"

% Atalho para inserir código NASM x86 no pdf.
% Uso: \putNASM{legenda}{nome do arquivo}{label para referência}
% O arquivo deve estar na pasta 'listings'
\usepackage{float}
\newcommand{\putNASM}[3]{
    \begin{listing}[H]
        \caption{#1}
        \inputminted[linenos=false, frame=single]{nasm}{listings/#2}
        \label{#3}
    \end{listing}
}

\newcommand{\putC}[3]{
    \begin{listing}[H]
        \caption{#1}
        \inputminted[linenos=false, frame=single]{C}{listings/#2}
        \label{#3}
    \end{listing}
}

\begin{document}

\chapter{Exercícios de Software Básico}
    \section{Módulo 1 -- Compiladores}
        \subsection{Questões Teóricas}
        \begin{enumerate}
    \item
    Abaixo estão listadas várias afirmativas incorretas.
    Justificando, identifique os erros, e corrija-os.
    \begin{itemize}
        \item [(a)]
        O formato .COM caracteriza-se por ter um endereço fixo (100H)
        para o ponto de entrada do programa.
        O cabeçalho de um arquivo nesse formato possui tamanho reduzido simplesmente
        composto pelos caracteres `MZ' e 
        a quantidade de segmentos de 64 kiB necessários para esse programa.

        \item [(b)]
        Bootstrap loader é um carregador especial, já que ele consiste em um programa
        armazenado completamente em um único setor conhecido como MBR.

        \item [(c)]
        A utilização de bibliotecas dinâmicas permite que um trecho de código 
        chamado por vários problemas possa ter uma única cópia em memória,
        e somente carregada ao ser executada: 
        o montador e o ligador portanto não necessitam serem informados 
        sobre o uso de uma biblioteca dinâmica.

        \item [(d)]
        O formato ELF é um formato de exclusivo de arquivos objeto complexo 
        que armazena tabelas com informações de realocação para o ligador.
        O formato PE (portable executable) é um formato feito em base no formato ELF.
    \end{itemize} 

    \item 
    Abaixo estão listadas várias afirmativas, as quais podem estar erradas.
    Identifique o(s) erro(s), justifique o por quê, e corrija-o.
    \begin{itemize}
        \item [(a)]
        Executável ligado com biblioteca estática 
        possui a vantagem de ser mais portável
        que o executável com biblioteca dinâmica.
        O programa também é carregado mais rápido em memória.
        Porém, é possível ter várias cópias da mesma biblioteca em memória.
        Por esse motivo os formatos de arquivos mais recentes 
        como ELF e PE não permitem ligação estática.

        \item [(b)]
        A utilização de bibliotecas dinâmicas com carregador estático 
        permite que um trecho de código chamado por vários programas 
        possa ter uma única cópia em memória, e somente carregada ao ser executada.
        O montador e o ligador portanto não necessitam 
        serem informados sobre o uso de uma biblioteca dinâmica. 
    \end{itemize}

    \item
    Descreva as informações contidas no MBR, indicando suas partes.
    Indique qual é o objetivo do setor MBR no processo de carregação do Sistema Operacional.
    Indique onde se encontra o MBR.
    Descreva por que o MBR foi substituído pelo GPT 
    em sistemas computacionais modernos.
\end{enumerate}


        \subsection{Questões Práticas}
        \begin{enumerate}

    \item
    Dado o programa abaixo,    
    \putNASM{macronacci.asm}

    \begin{itemize}
        \item [(a)]
        Mostre como ficaria a MNT e a MDT, 
        explicando o que são essas tabelas e para que são usadas..

        \item [(b)]
        Mostre a tabela de símbolos assumindo que foi utilizado
        o algoritmo de passagem única, escrevendo as listas de pendências.
        (note que é necessário primeiro resolver as macros).
    \end{itemize}    

    \item
    Dado o código abaixo em assembly inventado visto em sala de aula,
    mostre o arquivo objeto desse programa, chamado PROG1.
    Coloque T na frente de cada linha da parte de texto,
    e H na frente de cada linha de todos os headers (caso necessário).
    Em cada linha de header, indique textualmente o significado do conteúdo.
    Caso necessário, informação de realocação pode ser dada 
    utilizando os 2 formatos vistos em aula para arquivos reais.
    \putNASM{PROG1.asm}
    
    \item
    Utilizando a linguagem Assembly hipotética vista em sala de aula:
    \begin{itemize}
        \item [(a)]
        Faça um programa em assembly que receba 1 número (positivo) do usuário, 
        e verifique se é múltiplo de 3:
        se sim mostrar 1 na tela, caso contrário mostrar 0. 
        A parte de dados deve ir depois da parte de código.
        A verificação se o número é múltiplo ou não 
        deve ser feito utilizando uma macro que recebe como argumento 
        o endereço de memória onde foi salvo o número digitado pelo usuário. 
        O que tem de ser feito fora da macro 
        é a leitura do número e a impressão na tela da saída.

        \item [(b)]
        Personifique um montador de passagem única e realize a montagem do programa.
        Apresente a Tabela de Símbolos resultante 
        e o código máquina antes de resolver as referências pendentes
        (ou seja mostrar a tabela de símbolos e as listas de pendências).
        Não precisa apresentar o código montado nesta parte.
    \end{itemize}

    \item
    Utilizando a linguagem Assembly hipotética vista em sala de aula:
    \begin{itemize}
        \item [(a)]
        Faça um programa que receba um número inteiro (de 16 bits) do usuário
        e escreva na tela uma sequência de 1s e/ou 0s 
        sendo a representação binária do número indicado pelo usuário 
        (do bit menos significativo ao mais significativo).

        \item [(b)]
        Personifique um montador de passagem única e realize a montagem do programa.
        Apresente a Tabela de Símbolos resultante e o código máquina,
        ambos antes de resolver as referências pendentes.
    \end{itemize}

    \item
    Considere os módulos a seguir na Linguagem de Montagem Hipotética apresentada em sala.
    \begin{multicols}{2}
        \putNASM{modA.asm}\columnbreak
        \putNASM{modB.asm}
    \end{multicols}

    \begin{itemize}
        \item [(a)]
        Personifique um montador e monte os módulos como uma sequência de números inteiros. 
        Apresenta o código montado e as tabelas resultantes.

        \item [(b)]
        Personifique um ligador e combine os módulos em um único arquivo executável.
        Apresente o código ligado indicando os endereços absolutos e relativos,
        escrevendo no cabeçalho antes do código um mapa de bits mediante a diretiva R
        (ex.: R 0101010001). 
    \end{itemize}

    \item 
    Considere os módulos a seguir na Linguagem de Montagem Hipotética apresentada em sala.
    \begin{multicols}{3}
        \putNASM{moda.asm}\columnbreak
        \putNASM{modb.asm}\columnbreak
        \putNASM{modc.asm}
    \end{multicols}
    \begin{itemize}
        \item [(a)]
        Personifique um montador e monte os módulos. 
        Apresente as tabelas de símbolos, de uso e de definição de cada módulo
        (não é necessário apresentar o código montado de cada módulo).

        \item [(b)]
        Personifique um ligador e combine os módulos em um único arquivo executável.
        Somente é necessário apresentar o código ligado final indicando no código
        os endereços absolutos e relativos e os fatores de correção.
        O módulo A deve ir primeiro no código final, seguido de B e finalmente o C.
    \end{itemize}

    \item
    É preciso fazer um programa em Assembly formado por 2 módulos.
    O programa deve operar da seguinte forma:
    \begin{enumerate}
        \item [1.]
        O primeiro módulo deve pedir ao usuário 2 números.
        Deve armazenar esses dois dígitos em memória (rótulo)
        que foi reservada para receber os 2 valores.
        Ou seja, um só rótulo foi reservado para 2 endereços de memória.
        Esse rótulo deve ter sido reservado no módulo 2.

        \item [2.]
        Após receber e salvar os 2 números, o primeiro módulo 
        deve pular para o segundo módulo.

        \item [3.]
        O segundo módulo deve verificar se o segundo número é diferente de zero.
        Se for diferente de zero, deve voltar ao módulo 1.

        \item [4.]
        O módulo 1 deve então mostrar a divisão do primeiro pelo segundo.
    \end{enumerate}

    \begin{itemize}
        \item [(a)]
        Mostre os 2 módulos utilizando o Assembly inventado visto em sala de aula.
        Cada módulo sempre deve ter a seção de dados depois da seção de texto.
        Não é necessário utilizar a diretiva SECTION para dividir as seções,
        mas é obrigatório o uso de BEGIN e END.

        \item [(b)]
        Mostre a tabela de símbolos, uso e definições de cada módulo,
        assim como o fator de correção de cada módulo.

        \item [(c)]
        Mostre o código objeto após os módulos terem sido ligados.
    \end{itemize}

    \item
    Um programa em Assembly hipotético visto em sala de aula 
    recebe três inteiros positivos distintos 
    e realiza as seguintes operações sobre eles:
    \begin{itemize}
        \item [(i)] salva cada número numa label;
        \item [(ii)] identifica o maior e o menor;
        \item [(iii)] imprime na tela o menor, e depois o maior;
        \item [(iv)] repete o processo se o usuário der enter, encerra se não.
    \end{itemize}

    \begin{itemize}
        \item [(a)]
        Elabore o programa em 2 módulos.
        O módulo 1 deve fazer as tarefas (i) e (iv), 
        e o 2, (ii) e (iii).
        A seção de dados deve sempre ir ao final.

        \item [(b)]
        Mostre as tabelas de uso, símbolo
        e definições de cada um dos módulos.

        \item [(c)]
        Ligue o programa, mostrando 
        o código máquina do assembly inventado,
        os fatores de correção 
        e a tabela global de definições.
        O código máquina deve ser mostrado 
        seguindo o formato visto em aula.
    \end{itemize}

    \item
    Faça um programa que receba números do usuário.
    O primeiro número deve ser simplesmente salvo,
    e depois somado com o segundo número.
    O terceiro número será subtraído da soma precedente.
    O quarto número será somado ao montante,
    o quinto subtraído, e por aí vai, até ser atingida a condição de parada.
    \begin{itemize}
        \item
        O programa deve ser escrito em 2 módulos, 
        sempre com as diretivas \asm{SPACE} e \asm{CONST} ao final

        \item
        O primeiro módulo deve ler um número do usuário, que será a condição de parada.
        Esse valor deve ser salvo numa variável chamada STOP1 dentro do módulo.
        STOP2 deve guardar o oposto negativo desse valor.
        Assuma que o usuário digitará sempre valores positivos.

        \item
        O segundo módulo deve fazer o laço
        que pede ao usuário uma série de números,
        realizando as operações intercaladas descritas no enunciado,
        registrando a quantidade de somas e subtrações feitas.
        Quando o montante for maior ou igual a STOP1, ou menor que STOP2,
        o laço acaba, e o módulo deve pular para o primeiro.

        \item
        O primeiro módulo mostra a quantidade de somas/subtrações, e termina.
    \end{itemize}

    \begin{itemize}
        \item [(a)]
        Mostre o programa usando o Assembly inventado dos 2 módulos.

        \item [(b)]
        Mostre a tabela de símbolos, uso e definição de cada módulo.

        \item [(c)]
        Mostre o código de máquina ligado
        (não precisa indicar absolutos e relativos).
    \end{itemize}
\end{enumerate}


    \section{Módulo 2 -- Assembly x86-64}
        \subsection{Questões Teóricas}
        \begin{enumerate}
    \item
    Os itens abaixo possuem instruções de programas Assembly IA-32 (em modo nativo) 
    que utilizam diversos modos de endereçamento. 
    Classifique cada item como correto ou errado, e justifique o que estiver errado.
    \begin{itemize}
        \item[(a)] \asm{mov EAX, 10}
        \item[(b)] \asm{mov [M], AL}
        \item[(c)] \asm{mov AL, [CS + ESI + array]}
        \item[(d)] \asm{mov vetor[1], 0}
        \item[(e)] \asm{add AX, [X + ECX]}
        \item[(f)] \asm{mov ESI, vetor + EBX}
        \item[(g)] \asm{inc WORD [inicio + EBX*8 + ESI]}
        \item[(h)] \asm{mov [EBX + ESI*4], DWORD 5}
        \item[(i)] \asm{dec BYTE [BL]}
        \item[(j)] \asm{add [x + 1], AL} 
        \item[(k)] \asm{mov EAX, [array + ECX*8 + EBX]}
        \item[(l)] \asm{mov [EAX*8 + 1], 5}
        \item[(m)] \asm{mov BL, AX}
        \item[(n)] \asm{cmp [ESI], 10}
        \item[(o)] \asm{adc AL, AH}
    \end{itemize}

    \item
    Descreva as diferenças sobre endereçamento e utilização de meória 
    do Modo Real e o Modo Protegido. 
    Para o modo real, indique somente como é calculado o endereço de memória.
    No modo protegido, 
    faça um diagrama mostrando os diferentes segmentos de memória,
    indique como é calculado o endereço lógico e real, 
    e por que o modo é chamado de protegido.

    \item 
    Dadas as seguintes instruções e os seus respectivos códigos de máquina,
    indique os valores dos campos OPCODE, Mod R/M, SIB, DISPLACEMENT, e IMMEDIATE.
    Note que uma instrução pode deixar de apresentar algum campo.
    \begin{itemize}
        \item \asm{MOV edx, 0x0} ba 00 00 00 00
        \item \asm{MOV EBP, ESP} 89 e5
    \end{itemize}

    \item
    Existem três tipos básicos de operandos: imediato, registrador e memória.
    O acesso a memória pode ser feito de duas maneiras: direta ou indireta.
    Em cada instrução com algum tipo de endereçamento 
    do código abaixo especifique que tipo de operando 
    está sendo usado como fonte e destino:
    imediato, memória direta/indireta, ou registrador.
    Indicar se algum endereçamento é ilegal. 

    \putNASM{Código com vários tipos de acesso a memória.}{enderecamentos.asm}{enderecamentos}

\end{enumerate}

			
        \subsection{Questões Práticas}
        \begin{enumerate}[resume]
    \item
    O programa abaixo realiza a cópia de um vetor de \textit{double words}, 
    convertendo-o de \textit{little endian} para \textit{big endian}.
    Complete o programa, indicando as instruções dos espaços em branco 
    (cada espaço deve ser preenchido com uma única instrução).

    \putNASM{Conversor de \textit{little endian} para \textit{big endian}.}{little_endian.asm}{little_endian}

    \item
    Para cada código C abaixo, escreva o equivalente em Assembly IA-32.
    Diretivas em C \textbf{devem} ser substituídas por diretivas equivalentes em IA-32.
    Use os registradores para as variáveis locais (com exceção de estruturas de dados)
    e seção de Dados ou BSS para as variáveis estáticas ou globais.
    \textbf{Deve-se} utilizar os endereçamentos corretos 
    para cada tipo de estrutura de dados.
    Não se preocupe pelo fato do programa principal em C ser uma função.
    \begin{itemize}
        \item [(a)] \putC{Vetor de 100 inteiros.}{vector_size_100.c}{vector_size_100}
        \item [(b)] \putC{Matriz indicadora de inteiros iguais.}{matrix_matching.c}{matrix_matching}
        \item [(c)] \putC{Soma dos elementos de um vetor.}{vector_sum.c}{vector_sum}
        \item [(d)] Não é permitido \asm{MUL} ou \asm{IMUL}.
                    \putC{Preenchimento de matriz 100x100.}{100x100_matrix.c}{100x100_martix}
        \item [(e)] Assuma que o usuário vai digitar um número de 0 a 9.
                    \putC{Soma com parâmetro na entrada do usuário.}{weird_sum.c}{weird_sum}
    \end{itemize}
\end{enumerate}

		
%\chapter{Respostas aos exercícios}
%\section{Módulo 1 -- Compiladores}
%\subsection{Questões Teóricas}
%\subsection{Questões Práticas}

%\section{Módulo 2 -- Assembly x86-64}
%\subsection{Questões Teóricas}
%\subsection{Questões Práticas}

\end{document}
