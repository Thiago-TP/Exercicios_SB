% Pacotes padrões para lidar com palavras em português
\usepackage[T1]{fontenc}    % https://ctan.org/pkg/fontenc?lang=en
\usepackage[utf8]{inputenc} % https://ctan.org/pkg/inputenc?lang=en
\usepackage[brazil]{babel}  % https://ctan.org/pkg/babel?lang=en

% Borda uniforme de 2.5cm por todo o documento
\usepackage[margin=2.5cm]{geometry} % https://ctan.org/pkg/geometry?lang=en

% Decorações pela página
\usepackage{fancyhdr} % https://ctan.org/pkg/fancyhdr?lang=en
\usepackage{lastpage}
    \pagestyle{fancy}
    \renewcommand{\sectionmark}[1]{\markright{#1}}
    \renewcommand{\chaptermark}[1]{%
    \markboth{#1}{}}

    \renewcommand{\headrule}{\rule{\textwidth}{.5pt}}

    \fancyhead[L]{Módulo \thechapter\ -- \leftmark}
    \fancyhead[C]{}
    \fancyhead[R]{\rightmark}

    \fancyfoot[L]{Sotfware Básico}
    \fancyfoot[R]{\thepage\ de \pageref{LastPage}}

    \renewcommand{\footrule}{\rule{\textwidth}{.5pt}}

% Pacote para desenhos
\usepackage{tikz} % https://ctan.org/pkg/pgf?lang=en

% Pacoe para coluna dupla/tripla
\usepackage{multicol}

% Mudança da estética dos títulos de capítulo, seção, e a subseção
\usepackage[scaled]{helvet} % https://ctan.org/pkg/helvet?lang=en
\usepackage{titlesec}       % https://ctan.org/pkg/titlesec?lang=en
    \titleformat{\chapter}
    {\sffamily\huge\bfseries\color{cyan!60!black}}
    {Módulo \thechapter\ --}{5pt}{}

    \titleformat{\section}
    {\normalfont\sffamily\LARGE\bfseries\color{cyan!50!black}}
    {\thesection\ }{0pt}{}

% Pacote enumitem permite colocar número da seção na enumeração, e introduz a opção resume.
% https://tex.stackexchange.com/questions/1126/include-section-number-in-list-number
% https://tex.stackexchange.com/questions/351638/continue-numbering-of-paragraphs-after-a-new-section
\usepackage{enumitem} % https://ctan.org/pkg/enumitem?lang=en
    \setlength{\parindent}{0em}
    \setlist{noitemsep, leftmargin=0em}
    \setenumerate[1]{label=\thesection.\arabic*.}
    \setenumerate[2]{label*=\arabic*.}

% Atalho para escrever uma linha de código.
% Uso: \asm{linha de código}
% Na linha, os símbolos _ devem ser reescritos como \_
\newcommand{\asm}[1]{{\tt#1}}

% Pacote para inserção de códigos
\usepackage{minted} % https://ctan.org/pkg/minted?lang=en
    \usemintedstyle{lovelace} % caso queira mudar o esquema de cores do código, troque por uma das opções em "pygmentize -L styles"
    % Muda a cor, tamanho e fonte dos números que enumeram linhas de código
    \renewcommand{\theFancyVerbLine}{\textcolor[rgb]{.5, .5, .5}{\tiny\arabic{FancyVerbLine}}}
% Atalho para inserir código NASM, C e objdump no pdf.
% Uso: \putNASM{nome do arquivo.asm}
% O arquivo deve estar na pasta 'listings' !
\definecolor{bg}{rgb}{0.97, 0.97, 0.97} % cor do fundo dos códigos
\newcommand{\putNASM}[1]{
    \inputminted[numbers=right, numbersep=1pt, bgcolor=bg]{nasm}{listings/#1}
}
\newcommand{\putC}[1]{
    \inputminted[numbers=right, numbersep=1pt, bgcolor=bg]{C}{listings/#1}
}
\newcommand{\putDUMP}[1]{
    \inputminted[numbers=right, numbersep=1pt, bgcolor=bg]{text}{listings/#1}
}
