% Pacotes padrões para lidar com palavras em português
\usepackage[T1]{fontenc}    % https://ctan.org/pkg/fontenc?lang=en
\usepackage[utf8]{inputenc} % https://ctan.org/pkg/inputenc?lang=en
\usepackage[brazil]{babel}  % https://ctan.org/pkg/babel?lang=en

% Borda uniforme de 2.5cm por todo o documento
\usepackage[margin=2.5cm]{geometry} % https://ctan.org/pkg/geometry?lang=en

% Pacote para frações "deitadinhas"
\usepackage{xfrac}

% Decorações pela página
\usepackage{fancyhdr} % https://ctan.org/pkg/fancyhdr?lang=en
\usepackage{lastpage} % https://ctan.org/pkg/lastpage?lang=en
    \pagestyle{fancy}

    % \rightmark dá apenas o nome da seção:
    % https://tex.stackexchange.com/questions/310260/retrieve-only-chapter-name-in-fancyhdr
    \renewcommand{\sectionmark}[1]{\markright{#1}}

    % \leftmark dá apenas o nome do capítulo:
    % https://tex.stackexchange.com/questions/310260/retrieve-only-chapter-name-in-fancyhdr
    \renewcommand{\chaptermark}[1]{%
    \markboth{#1}{}}

    % linha fina entre as decorações no cabeçalho e o texto
    \renewcommand{\headrule}{\rule{\textwidth}{.5pt}}

    % Decorações no cabeçalho
    \fancyhead[L]{Módulo \thechapter\ -- \leftmark} % à esquerda: "Módulo X - Nome do módulo"
    \fancyhead[C]{}                                 % ao centro: nada
    \fancyhead[R]{\rightmark}                       % à direita: "Nome da seção"

    % Decorações no rodapé
    \fancyfoot[L]{Sotfware Básico}                  % à esquerda: "Software Básico"
                                                    % ao centro: "Lista de Questões/Respostas", a ser definido no .tex referente
    \fancyfoot[R]{pg. \thepage\ de \pageref{LastPage}}  % à direita: "X de Y"

    % linha fina entre as decorações do rodapé e o texto
    \renewcommand{\footrule}{\rule{\textwidth}{.5pt}}

% Pacote para colorir linhas de tabela
% https://tex.stackexchange.com/questions/112671/define-and-use-new-colors-for-rowcolor-colortbl
\usepackage[table]{xcolor} % https://ctan.org/pkg/xcolor?lang=en

% Pacote para desenhos
\usepackage{tikz} % https://ctan.org/pkg/pgf?lang=en

% Pacote para coluna dupla/tripla
\usepackage{multicol} % https://ctan.org/pkg/multicol?lang=en

% Mudança da estética dos títulos de capítulo, seção, e a subseção
\usepackage[scaled]{helvet} % https://ctan.org/pkg/helvet?lang=en
\usepackage{titlesec}       % https://ctan.org/pkg/titlesec?lang=en
    \titleformat{\chapter}
    {\sffamily\huge\bfseries\color{cyan!60!black}}
    {Módulo \thechapter\ --}{5pt}{}

    \titleformat{\section}
    {\normalfont\sffamily\LARGE\bfseries\color{cyan!50!black}}
    {\thesection\ }{0pt}{}

% Pacote enumitem permite colocar número da seção na enumeração, e introduz a opção resume.
% https://tex.stackexchange.com/questions/1126/include-section-number-in-list-number
% https://tex.stackexchange.com/questions/351638/continue-numbering-of-paragraphs-after-a-new-section
\usepackage{enumitem} % https://ctan.org/pkg/enumitem?lang=en
    \setlength{\parindent}{0em}
    \setlist{noitemsep, leftmargin=0em}
    \setenumerate[1]{label=\thesection.\arabic*.}
    \setenumerate[2]{label*=\arabic*.}

% Atalho para escrever uma linha de código.
% Uso: \asm{linha de código}
% Na linha, os símbolos _ devem ser reescritos como \_
\newcommand{\asm}[1]{{\tt#1}}

% Pacote para inserção de códigos
\usepackage{minted} % https://ctan.org/pkg/minted?lang=en
    % Caso queira mudar o esquema de cores do código, 
    % troque por uma das opções em "pygmentize -L styles"
    \usemintedstyle{lovelace} 
    
    % Muda a cor, tamanho e fonte dos números que enumeram linhas de código
    % https://tex.stackexchange.com/questions/132849/how-can-i-change-the-font-size-of-the-number-in-minted-environment
    \renewcommand{\theFancyVerbLine}{\textcolor[rgb]{.5, .5, .5}{\tiny\arabic{FancyVerbLine}}}

    % Atalhos para inserir código NASM, C e objdump no pdf.
    % Uso: \putNASM{nome do arquivo.asm}
    % Coloque o arquivo deve estar na pasta "listings" !
    \definecolor{bg}{rgb}{0.95, 0.95, 0.95} % cor do fundo dos códigos
    \newcommand{\putNASM}[1]{\inputminted[numbers=right, numbersep=1pt, bgcolor=bg]{nasm}{listings/#1}}
    \newcommand{\putC}[1]   {\inputminted[numbers=right, numbersep=1pt, bgcolor=bg]{C}{listings/#1}}
    \newcommand{\putDUMP}[1]{\inputminted[numbers=right, numbersep=1pt, bgcolor=bg]{text}{listings/#1}}
