\begin{enumerate}
    \item
    \putNASM{respostas/little_endian_complete.asm}

    \item $\!$
    \begin{itemize}
        \item [(a)] \putNASM{respostas/vector_size_MAX.asm}
        \item [(b)] \putNASM{respostas/matrix_matching.asm}
        \item [(c)] \putNASM{respostas/vetor_SIZE.asm}
        \item [(d)] \putNASM{respostas/100x100_matrix.asm}
        \item [(e)] \putNASM{respostas/weird_sum.asm}
    \end{itemize}
    
    \item $\!$
    \begin{itemize}
        \item [(a)] \putC{respostas/foo1.c}
        \item [(b)] \putC{respostas/foo2.c}
        \item [(c)] \putC{respostas/foo3.c}
        \item [(d)] \putC{respostas/foo4.c}
    \end{itemize}

    \item 
    \putNASM{respostas/f4_mul_elements.asm}

    \item
    \putNASM{respostas/f4_mul_vectors.asm}

    \item
    Modificações necessárias no código C original:
    além de eliminar a definição original da função,
    deve também declarar a assinatura da \asm{soma} em Assembly por
    \asm{extern void soma(int *M, int N, int *valor)}.
    \putNASM{respostas/soma.asm}

    \item
    \putNASM{respostas/f1.asm}

    \item
    \putNASM{respostas/files.asm}

    \item
    \putNASM{respostas/files_sum.asm}

    \item
    Note que os arrays/buffers $x$ e $y$ têm 200 bytes de conteúdo, e não 100.
    \putNASM{respostas/files_0s_and_1s.asm}
    \item
    O programa a seguir multiplica matrizes
    de tamanhos arbitrários e compatíveis.
    O procedimento auxiliar \asm{GetMat}
    realiza os laços de preenchimento das matrizes.
    \putNASM{respostas/mul_mat_5x5.asm}

    \item Bias $= 011 = 3$, Regra = {\color{red}\textit{ceil}}. 
    \begin{table}[H]
        \begin{tabular}{|l|l|l|l|l|}
            \hline
            \textbf{Descrição}  &
            \textbf{Binário}    &
            \textbf{Mantissa}   &
            \textbf{Expoente}   &
            \textbf{Valor decimal} \\\hline
            Menos zero 
            & 1 000 00 & $0.0$ & $-2$ & $-0.0$
            \\\hline
            Número positivo mais próximo a zero
            & 0 000 01 & $\sfrac{1}{4}$ & $-2$ & $\sfrac{1}{4}\times 2^{-2}$
            \\\hline
            Infinito negativo
            & 1 111 00 & -  & - & - 
            \\\hline
            Maior número normalizado
            & 0 110 11 & $1\sfrac{3}{4}$ & $3$ & $1\sfrac{3}{4}\times 2^3$
            \\\hline
            Menor número não-normalizado
            & 1 000 11 & $\sfrac{3}{4}$ & $-2$ & $-\sfrac{3}{4}\times 2^{-2}$
            \\\hline\rowcolor{red!25}
            $5.0 - 0.75 = 4.25$
            & 0 101 01 & $1\sfrac{1}{4}$ & $2$ & $1\sfrac{1}{4}\times 2^2 = 5.0$
            \\\hline
            $4.0 + 3.0 = 7.0$
            & 0 101 11 & $1\sfrac{3}{4}$ & $2$ & $1\sfrac{3}{4}\times 2^{2} = 7.0$
            \\\hline
        \end{tabular}
    \end{table}

    \item Bias $= 0111 = 7$, Regra = {\color{red}\textit{round}}
    \begin{table}[H]
        \begin{tabular}{|l|l|l|l|l|}
            \hline
            \textbf{Descrição}  &
            \textbf{Binário}    &
            \textbf{Mantissa}   &
            \textbf{Expoente}   &
            \textbf{Valor decimal} \\\hline
            Menos zero
            & 1 0000 00 & 0 & $-2$ & $-0.0$
            \\\hline
            Número positivo mais próximo a zero
            & 0 0000 01 & $\sfrac{1}{4}$ & $-6$ & $\sfrac{1}{4}\times 2^{-6}$
            \\\hline
            Maior número normalizado
            & 0 1110 11 & $1\sfrac{3}{4}$ & $7$ & $1\sfrac{3}{4}\times 2^7$
            \\\hline
            Menor número não-normalizado
            & 1 0000 11 & $\sfrac{3}{4}$ & $-6$ & $-\sfrac{3}{4}\times 2^{-6}$
            \\\hline
            $4.0 + 3.0 = 7.0$
            & 0 1001 11 & $1\sfrac{3}{4}$ & $2$ & $1\sfrac{3}{4}\times 2^2 = 7.0$
            \\\hline\rowcolor{red!25}
            $7.0 + 8.0 =  15.0$
            & 0 1010 11 & $1\sfrac{3}{4}$ & $3$ & $1\sfrac{3}{4}\times 2^3 = 14.0$
            \\\hline
        \end{tabular}
    \end{table}

    \item Bias $= 0111 = 7$, Regra = {\color{red}fração mais próxima}.
    \begin{table}[H]
        \begin{tabular}{|l|l|l|l|l|}
            \hline
            \textbf{Número}  &
            \textbf{Valor}    &
            \textbf{Bit sinal}   &
            \textbf{Bits expoente}   &
            \textbf{Bits mantissa} \\\hline
            Zero
            & $0.0$ & 0 & 0000 & 0000
            \\\hline
            Negativo mais próximo a zero
            & $-\sfrac{1}{16}\times 2^{-6}$ & 1 & 0000 & 0001
            \\\hline
            Maior positivo
            & $1\sfrac{15}{16}\times 2^7$ & 0 & 1110 & 1111
            \\\hline
            n/a & $-5.0$
            & 1 & 1001 & 0100
            \\\hline
            n/a & $1\sfrac{9}{16}\times2^{-2}$ 
            & 0 & 0101 & 1001
            \\\hline
            Menos um & $-1.0$
            & 1 & 0111 & 0000
            \\\hline\rowcolor{red!25}
            $4 - 1\sfrac{9}{16} = \sfrac{39}{16} = 2.4375$
            & $\sfrac{40}{16} = 2.5$ & 0 & 1000 & 0100 
            \\\hline
        \end{tabular}
    \end{table}

    \item Bias $= 01111 = 15$, Regra = {\color{red}par mais próximo}.
    \begin{table}[H]
        \begin{tabular}{|l|l|l|l|l|}
            \hline
            \textbf{Descrição}  &
            \textbf{Binário}    &
            \textbf{Mantissa}   &
            \textbf{Expoente}   &
            \textbf{Valor decimal} \\\hline
            Número negativo mais próximo a zero
            & 1 00000 0001 & $\sfrac{1}{16}$ & $-14$ & $-\sfrac{1}{16}\times 2^{-14}$
            \\\hline
            Maior número
            & 0 11110 1111 & $1\sfrac{15}{16}$ & $15$ & $1\sfrac{15}{16}\times 2^{15}$
            \\\hline
            Menor número não-normalizado
            & 1 00000 1111 & $\sfrac{15}{16}$ & $-14$ & $\sfrac{15}{16}\times 2^{-14}$
            \\\hline
            Menos um
            & 1 01111 0000 & $1.0$ & $0$ & $-1.0$
            \\\hline
        \end{tabular}
    \end{table}

\end{enumerate}
