\begin{enumerate}
    \item
    As tabelas, assim como o código com as macros resolvidas, são apresentadas abaixo.
    O código máquina foi deduzido para fins de debug no simulador.
    \putDUMP{respostas/macronacci_tabela.asm}

    
    \item
    \putDUMP{respostas/PROG1_ligado.asm}

    \item
    \putDUMP{respostas/multiplo_de_3.asm}

    \item 
    \putDUMP{respostas/write_binary.asm} 
    
    \item
    Os códigos de cada módulo são apresentados a seguir.
    \putDUMP{respostas/modA_tabelas.asm}
    \putDUMP{respostas/modB_tabelas.asm}
    
    Por fim, apresenta-se o código montado com o mapa de bits.
    \putDUMP{respostas/modAB_ligado.asm}

    \item
    Os códigos de cada módulo são apresentados a seguir.
    Para facilitar a ligação, é mostrado 
    o código máquina não-ligado de cada um.
    \putDUMP{respostas/moda_tabelas.asm}
    \putDUMP{respostas/modb_tabelas.asm}
    \putDUMP{respostas/modc_tabelas.asm}

    A seguir, apresenta-se o código máquina dos módulos ligados.
    \putDUMP{respostas/modabc_ligado.asm}

    \item
    Os códigos de cada módulo são apresentados a seguir.
    Para facilitar a ligação, é mostrado 
    o código máquina não-ligado de cada um.
    \putDUMP{respostas/mod1_calcdiv.asm}
    \putDUMP{respostas/mod2_calcdiv.asm}

    A seguir, apresenta-se o código máquina dos módulos ligados,
    de forma que os códigos objeto de cada um estão sobrepostos.
    \putDUMP{respostas/calcdiv_ligado.asm}

    \item
    Os códigos de cada módulo são apresentados a seguir.
    Para facilitar a ligação, é mostrado 
    o código máquina não ligado de cada um.
    \putDUMP{respostas/mod1_sort.asm}
    \putDUMP{respostas/mod2_sort.asm}

    A seguir, apresenta-se o código máquina dos módulos ligados.
    \putDUMP{respostas/sort_ligado.asm}

    \item
\end{enumerate}