\begin{enumerate}
    \item
    Os itens abaixo possuem instruções de programas Assembly IA-32 (em modo nativo) 
    que utilizam diversos modos de endereçamento. 
    Classifique cada item como correto ou errado, e justifique o que estiver errado.
    \begin{itemize}
        \item[(a)] \asm{mov EAX, 10}
        \item[(b)] \asm{mov [M], AL}
        \item[(c)] \asm{mov AL, [CS + ESI + array]}
        \item[(d)] \asm{mov vetor[1], 0}
        \item[(e)] \asm{add AX, [X + ECX]}
        \item[(f)] \asm{mov ESI, vetor + EBX}
        \item[(g)] \asm{inc WORD [inicio + EBX*8 + ESI]}
        \item[(h)] \asm{mov [EBX + ESI*4], DWORD 5}
        \item[(i)] \asm{dec BYTE [BL]}
        \item[(j)] \asm{add [x + 1], AL} 
        \item[(k)] \asm{mov EAX, [array + ECX*8 + EBX]}
        \item[(l)] \asm{mov [EAX*8 + 1], 5}
        \item[(m)] \asm{mov BL, AX}
        \item[(n)] \asm{cmp [ESI], 10}
        \item[(o)] \asm{adc AL, AH}
    \end{itemize}

    \item
    Descreva as diferenças sobre endereçamento e utilização de meória 
    do Modo Real e o Modo Protegido. 
    Para o modo real, indique somente como é calculado o endereço de memória.
    No modo protegido, 
    faça um diagrama mostrando os diferentes segmentos de memória,
    indique como é calculado o endereço lógico e real, 
    e por que o modo é chamado de protegido.

    \item 
    Dadas as seguintes instruções e os seus respectivos códigos de máquina,
    indique os valores dos campos OPCODE, Mod R/M, SIB, DISPLACEMENT, e IMMEDIATE.
    Note que uma instrução pode deixar de apresentar algum campo.
    \begin{itemize}
        \item \asm{MOV edx, 0x0} ba 00 00 00 00
        \item \asm{MOV EBP, ESP} 89 e5
    \end{itemize}

    \item
    Existem três tipos básicos de operandos: imediato, registrador e memória.
    O acesso a memória pode ser feito de duas maneiras: direta ou indireta.
    Em cada instrução com algum tipo de endereçamento 
    do código abaixo especifique que tipo de operando 
    está sendo usado como fonte e destino:
    imediato, memória direta/indireta, ou registrador.
    Indicar se algum endereçamento é ilegal. 

    \putNASM{Código com vários tipos de acesso a memória.}{enderecamentos.asm}{enderecamentos}

\end{enumerate}
