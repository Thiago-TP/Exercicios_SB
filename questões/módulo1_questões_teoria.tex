\begin{enumerate}

    \item 
    Responda sucintamente:
    \begin{itemize}
        \item [(a)]
        Qual a diferença mais importante entre uma macro 
        e uma subrotina (função)?

        \item [(b)]
        Durante o processo de tradução são necessários dois estágios,
        Análise e Síntese. Explique brevemente o objetivo de cada estágio
        e liste as sub-etapas de cada um deles.        
    \end{itemize}

    \item 
    Os itens abaixo podem conter erros.
    Destaque-os, e corrija-os reescrevendo os itens.
    \begin{itemize}
        \item [(a)]
        O bootstrap loader faz parte do Sistema Operacional.
        Ele é carregado no processo de BOOT.
        Possui um tamanho sempre múltiplo a 512 bytes 
        e foi substituído pela GPT para ocupar muito espaço em memória RAM.

        \item [(b)]
        As bibliotecas dinâmicas com carregador estático 
        geram um executável chamado de standalone que tem como vantagem
        ser mais portável que o executável gerado usando 
        bibliotecas dinâmicas com carregador dinâmico.
        Porém, o carregador dinâmico é o único que garante que 
        somente tenha uma única cópia da biblioteca em memória.

        \item [(c)]
        O formato de arquivos .COM é um formato sem cabeçalho 
        de no máximo 64 kB que não permite ligação nem debug.
        É um arquivo tanto para objeto como executável.
        O formato COFF também é formato de objeto executável,
        porém ele permite bibliotecas dinâmicas e debug.

        \item [(d)]
        A principal vantagem de uma linguagem compilada em relação
        a uma interpretada é a otimização completa do código 
        sem precisar manter relação de linha de código compilado
        com o arquivo de texto de entrada com o programa original.
        Porém, uma outra vantagem é que os programas compilados
        são mais portáveis, já que o arquivo executável de um programa
        compilado pode ser executado em qualquer Sistema Operacional.
    \end{itemize}

    \item
    Abaixo estão listadas várias afirmativas incorretas.
    Justificando, identifique os erros, e corrija-os.
    \begin{itemize}
        \item [(a)]
        O formato .COM caracteriza-se por ter um endereço fixo (100H)
        para o ponto de entrada do programa.
        O cabeçalho de um arquivo nesse formato possui tamanho reduzido simplesmente
        composto pelos caracteres `MZ' e 
        a quantidade de segmentos de 64 kiB necessários para esse programa.

        \item [(b)]
        Bootstrap loader é um carregador especial, já que ele consiste em um programa
        armazenado completamente em um único setor conhecido como MBR.

        \item [(c)]
        A utilização de bibliotecas dinâmicas permite que um trecho de código 
        chamado por vários problemas possa ter uma única cópia em memória,
        e somente carregada ao ser executada: 
        o montador e o ligador portanto não necessitam serem informados 
        sobre o uso de uma biblioteca dinâmica.

        \item [(d)]
        O formato ELF é um formato de exclusivo de arquivos objeto complexo 
        que armazena tabelas com informações de realocação para o ligador.
        O formato PE (portable executable) é um formato feito em base no formato ELF.
    \end{itemize} 

    \item 
    Abaixo estão listadas várias afirmativas, as quais podem estar erradas.
    Identifique o(s) erro(s), dê o por quê(s), e corrija-o(s).
    \begin{itemize}
        \item [(a)]
        Executável ligado com biblioteca estática 
        possui a vantagem de ser mais portável
        que o executável com biblioteca dinâmica.
        O programa também é carregado mais rápido em memória.
        Porém, é possível ter várias cópias da mesma biblioteca em memória.
        Por esse motivo os formatos de arquivos mais recentes 
        como ELF e PE não permitem ligação estática.

        \item [(b)]
        A utilização de bibliotecas dinâmicas com carregador estático 
        permite que um trecho de código chamado por vários programas 
        possa ter uma única cópia em memória, e somente carregada ao ser executada.
        O montador e o ligador portanto não necessitam 
        serem informados sobre o uso de uma biblioteca dinâmica. 
    \end{itemize}

    \item
    Descreva as informações contidas no MBR, indicando suas partes.
    Indique qual é o objetivo do setor MBR no processo de carregação do Sistema Operacional.
    Indique onde se encontra o MBR.
    Descreva por que o MBR foi substituído pelo GPT 
    em sistemas computacionais modernos.

    \item
    Explique para que serve o código de 3 endereços
    durante o processo de compilação;
    se é usado durante a fase de análise ou síntese e;
    qual a diferença entre a tabela de 3 colunas e a de 4.

    \item
    Dado o código de três endereços abaixo, 
    que trabalha com array de bytes em memória,
    responda:    
    \putC{3_addresses.c}

    \begin{itemize}
        \item [(a)]
        O código foi otimizado? Justifique.
        
        \item [(b)]
        Assumindo que o código inicial era ANSI C, 
        qual o tipo da variável c2 e
        qual o tipo da variável c1? Jusifique.         
    \end{itemize}

    \item
    Cada um dos seguintes trechos de código em linguagem C 
    possui um ou mais erros. Indique onde estão os erros 
    e classifique-os como léxico, sintático ou semântico.
    \begin{itemize}
        \item [(a)] \putC{for_ab.c}
        \item [(b)] \putC{double_xy.c}
        \item [(c)] \putC{switch_a.c}
        \item [(d)] \putC{float_abc.c}
    \end{itemize}

    \item
    As funções em C abaixo fazem parte do mesmo programa,
    como um arquivo de cabeçalho (*.h). O arquivo não compila.
    Indique as linhas erradas, explicitando se o erro é
    léxico, sintático ou semântico.
    Assuma que CHAR\_BIT foi definido corretamente.
    \putC{h_funcs.c}

    \item
    Detecte os erros no código abaixo, sublinhando o erro
    e indicando se o erro é sintático, léxico ou semântico.
    \putC{print_plus_n.c}

    \item
    No código abaixo identifique os erros
    e indique se é erro léxico, semântico ou sintático.
    \putC{mnpq.c}

    \item
    Indique quais são os erros no código abaixo, classificando-o como
    léxico, sintático ou semântico.
    \putC{wrong_array.c}

\end{enumerate}
