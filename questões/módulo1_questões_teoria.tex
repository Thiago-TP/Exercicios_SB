\begin{enumerate}
    \item
    Abaixo estão listadas várias afirmativas incorretas.
    Justificando, identifique os erros, e corrija-os.
    \begin{itemize}
        \item [(a)]
        O formato .COM caracteriza-se por ter um endereço fixo (100H)
        para o ponto de entrada do programa.
        O cabeçalho de um arquivo nesse formato possui tamanho reduzido simplesmente
        composto pelos caracteres `MZ' e 
        a quantidade de segmentos de 64 kiB necessários para esse programa.

        \item [(b)]
        Bootstrap loader é um carregador especial, já que ele consiste em um programa
        armazenado completamente em um único setor conhecido como MBR.

        \item [(c)]
        A utilização de bibliotecas dinâmicas permite que um trecho de código 
        chamado por vários problemas possa ter uma única cópia em memória,
        e somente carregada ao ser executada: 
        o montador e o ligador portanto não necessitam serem informados 
        sobre o uso de uma biblioteca dinâmica.

        \item [(d)]
        O formato ELF é um formato de exclusivo de arquivos objeto complexo 
        que armazena tabelas com informações de realocação para o ligador.
        O formato PE (portable executable) é um formato feito em base no formato ELF.
    \end{itemize} 

    \item 
    Abaixo estão listadas várias afirmativas, as quais podem estar erradas.
    Identifique o(s) erro(s), justifique o por quê, e corrija-o.
    \begin{itemize}
        \item [(a)]
        Executável ligado com biblioteca estática 
        possui a vantagem de ser mais portável
        que o executável com biblioteca dinâmica.
        O programa também é carregado mais rápido em memória.
        Porém, é possível ter várias cópias da mesma biblioteca em memória.
        Por esse motivo os formatos de arquivos mais recentes 
        como ELF e PE não permitem ligação estática.

        \item [(b)]
        A utilização de bibliotecas dinâmicas com carregador estático 
        permite que um trecho de código chamado por vários programas 
        possa ter uma única cópia em memória, e somente carregada ao ser executada.
        O montador e o ligador portanto não necessitam 
        serem informados sobre o uso de uma biblioteca dinâmica. 
    \end{itemize}

    \item
    Descreva as informações contidas no MBR, indicando suas partes.
    Indique qual é o objetivo do setor MBR no processo de carregação do Sistema Operacional.
    Indique onde se encontra o MBR.
    Descreva por que o MBR foi substituído pelo GPT 
    em sistemas computacionais modernos.
\end{enumerate}
