\begin{enumerate}

    \item
    Dado o programa abaixo,    
    \putNASM{macronacci.asm}

    \begin{itemize}
        \item [(a)]
        Mostre como ficaria a MNT e a MDT, 
        explicando o que são essas tabelas e para que são usadas..

        \item [(b)]
        Mostre a tabela de símbolos assumindo que foi utilizado
        o algoritmo de passagem única, escrevendo as listas de pendências.
        (note que é necessário primeiro resolver as macros).
    \end{itemize}    

    \item
    Dado o código abaixo em assembly inventado visto em sala de aula,
    mostre o arquivo objeto desse programa, chamado PROG1.
    Coloque T na frente de cada linha da parte de texto,
    e H na frente de cada linha de todos os headers (caso necessário).
    Em cada linha de header, indique textualmente o significado do conteúdo.
    Caso necessário, informação de realocação pode ser dada 
    utilizando os 2 formatos vistos em aula para arquivos reais.
    \putNASM{PROG1.asm}
    
    \item
    Utilizando a linguagem Assembly hipotética vista em sala de aula:
    \begin{itemize}
        \item [(a)]
        Faça um programa em assembly que receba 1 número (positivo) do usuário, 
        e verifique se é múltiplo de 3:
        se sim mostrar 1 na tela, caso contrário mostrar 0. 
        A parte de dados deve ir depois da parte de código.
        A verificação se o número é múltiplo ou não 
        deve ser feito utilizando uma macro que recebe como argumento 
        o endereço de memória onde foi salvo o número digitado pelo usuário. 
        O que tem de ser feito fora da macro 
        é a leitura do número e a impressão na tela da saída.

        \item [(b)]
        Personifique um montador de passagem única e realize a montagem do programa.
        Apresente a Tabela de Símbolos resultante 
        e o código máquina antes de resolver as referências pendentes
        (ou seja mostrar a tabela de símbolos e as listas de pendências).
        Não precisa apresentar o código montado nesta parte.
    \end{itemize}

    \item
    Utilizando a linguagem Assembly hipotética vista em sala de aula:
    \begin{itemize}
        \item [(a)]
        Faça um programa que receba um número inteiro (de 16 bits) do usuário
        e escreva na tela uma sequência de 1s e/ou 0s 
        sendo a representação binária do número indicado pelo usuário 
        (do bit menos significativo ao mais significativo).

        \item [(b)]
        Personifique um montador de passagem única e realize a montagem do programa.
        Apresente a Tabela de Símbolos resultante e o código máquina,
        ambos antes de resolver as referências pendentes.
    \end{itemize}

    \item
    Considere os módulos a seguir na Linguagem de Montagem Hipotética apresentada em sala.
    \begin{multicols}{2}
        \putNASM{modA.asm}\columnbreak
        \putNASM{modB.asm}
    \end{multicols}

    \begin{itemize}
        \item [(a)]
        Personifique um montador e monte os módulos como uma sequência de números inteiros. 
        Apresente o código montado e as tabelas resultantes.

        \item [(b)]
        Personifique um ligador e combine os módulos em um único arquivo executável.
        Apresente o código ligado indicando os endereços absolutos e relativos,
        escrevendo no cabeçalho antes do código um mapa de bits mediante a diretiva R
        (ex.: R 0101010001). 
    \end{itemize}

    \item 
    Considere os módulos a seguir na Linguagem de Montagem Hipotética apresentada em sala.
    \begin{multicols}{3}
        \putNASM{moda.asm}\columnbreak
        \putNASM{modb.asm}\columnbreak
        \putNASM{modc.asm}
    \end{multicols}
    \begin{itemize}
        \item [(a)]
        Personifique um montador e monte os módulos. 
        Apresente as tabelas de símbolos, de uso e de definição de cada módulo
        (não é necessário apresentar o código montado de cada módulo).

        \item [(b)]
        Personifique um ligador e combine os módulos em um único arquivo executável.
        Somente é necessário apresentar o código ligado final indicando no código
        os endereços absolutos e relativos e os fatores de correção.
        O módulo A deve ir primeiro no código final, seguido de B e finalmente o C.
    \end{itemize}

    \item
    É preciso fazer um programa em Assembly formado por 2 módulos.
    O programa deve operar da seguinte forma:
    \begin{enumerate}
        \item [1.]
        O primeiro módulo deve pedir ao usuário 2 números.
        Deve armazenar esses dois dígitos em memória (rótulo)
        que foi reservada para receber os 2 valores.
        Ou seja, um só rótulo foi reservado para 2 endereços de memória.
        Esse rótulo deve ter sido reservado no módulo 2.

        \item [2.]
        Após receber e salvar os 2 números, o primeiro módulo 
        deve pular para o segundo módulo.

        \item [3.]
        O segundo módulo deve verificar se o segundo número é diferente de zero.
        Se for diferente de zero, deve voltar ao módulo 1.

        \item [4.]
        O módulo 1 deve então mostrar a divisão do primeiro pelo segundo.
    \end{enumerate}

    \begin{itemize}
        \item [(a)]
        Mostre os 2 módulos utilizando o Assembly inventado visto em sala de aula.
        Cada módulo sempre deve ter a seção de dados depois da seção de texto.
        Não é necessário utilizar a diretiva SECTION para dividir as seções,
        mas é obrigatório o uso de BEGIN e END.

        \item [(b)]
        Mostre a tabela de símbolos, uso e definições de cada módulo,
        assim como o fator de correção de cada módulo.

        \item [(c)]
        Mostre o código objeto após os módulos terem sido ligados.
    \end{itemize}

    \item
    Um programa em Assembly hipotético visto em sala de aula 
    recebe três inteiros positivos distintos 
    e realiza as seguintes operações sobre eles:
    \begin{itemize}
        \item [(i)] salva cada número numa label;
        \item [(ii)] identifica o maior e o menor;
        \item [(iii)] imprime na tela o menor, e depois o maior;
        \item [(iv)] repete o processo se o usuário der enter, encerra se não.
    \end{itemize}

    \begin{itemize}
        \item [(a)]
        Elabore o programa em 2 módulos.
        O módulo 1 deve fazer as tarefas (i) e (iv), 
        e o 2, (ii) e (iii).
        A seção de dados deve sempre ir ao final.

        \item [(b)]
        Mostre as tabelas de uso, símbolo
        e definições de cada um dos módulos.

        \item [(c)]
        Ligue o programa, mostrando 
        o código máquina do assembly inventado,
        os fatores de correção 
        e a tabela global de definições.
        O código máquina deve ser mostrado 
        seguindo o formato visto em aula.
    \end{itemize}

    \item
    Faça um programa que receba números do usuário.
    O primeiro número deve ser simplesmente salvo,
    e depois somado com o segundo número.
    O terceiro número será subtraído da soma precedente.
    O quarto número será somado ao montante,
    o quinto subtraído, e por aí vai, até ser atingida a condição de parada.
    \begin{itemize}
        \item
        O programa deve ser escrito em 2 módulos, 
        sempre com as diretivas \asm{SPACE} e \asm{CONST} ao final

        \item
        O primeiro módulo deve ler um número do usuário, que será a condição de parada.
        Esse valor deve ser salvo numa variável chamada STOP1 dentro do módulo.
        STOP2 deve guardar o oposto negativo desse valor.
        Assuma que o usuário digitará sempre valores positivos.

        \item
        O segundo módulo deve fazer o laço
        que pede ao usuário uma série de números,
        realizando as operações intercaladas descritas no enunciado,
        registrando a quantidade de somas e subtrações feitas.
        Quando o montante for maior ou igual a STOP1, ou menor que STOP2,
        o laço acaba, e o módulo deve pular para o primeiro.

        \item
        O primeiro módulo mostra a quantidade de somas/subtrações, e termina.
    \end{itemize}

    \begin{itemize}
        \item [(a)]
        Mostre o programa usando o Assembly inventado dos 2 módulos.

        \item [(b)]
        Mostre a tabela de símbolos, uso e definição de cada módulo.

        \item [(c)]
        Mostre o código de máquina ligado
        (não precisa indicar absolutos e relativos).
    \end{itemize}
\end{enumerate}
